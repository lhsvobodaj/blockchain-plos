\keyword{publicação científica, \textit{open access}, \textit{peer review}, \textit{blockchain}}

\begin{abstract}
A liberdade para criação e acesso à informação evoluiu drasticamente com o advento da Web. Embora essa evolução possa ser observada em todas as áreas de conhecimento, o acesso ao conteúdo científico, por vezes, ainda está sob controle de algumas organizações. Uma reação a esse controle são as iniciativas \textit{open access} que fornecem acesso livre ao conteúdo publicado. A opção por publicações abertas levanta a discussão sobre como manter os direitos autorais sobre o trabalho, como efetuar um processo de revisão aberto e justo, e como recompensar os autores pelo seu trabalho, baseado no nível de colaboração de cada indivíduo. Desta forma, este trabalho tem como objetivo o desenvolvimento de uma plataforma aberta e colaborativa para escrita e revisão de trabalhos científicos, utilizando como base para isso a tecnologia de \textit{blockchain}. O uso de \textit{blockchain} para este fim traz consigo os benefícios inerentes a sua utilização, tais como durabilidade, confiabilidade e longevidade, já que os dados inseridos na \textit{chain} não podem ser alterados. Outro ponto importante da utilização desta tecnologia é o seu caráter de descentralização, o que facilita a contribuição entre os participantes e evita o controle sobre as informações armazenadas.
\end{abstract}