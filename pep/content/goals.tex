\chapter{Objetivos e Resultados Esperados}
O objetivo deste trabalho é a elaboração de uma proposta de plataforma aberta e colaborativa para criação e revisão de artigos científicos. Mais especificamente, a definição de contratos inteligentes que descrevam as interações necessárias contemplando ações de elaboração, revisão e acesso aos artigos desenvolvidos, bem como a especificação de um modelo de negócio capaz se sustentar a plataforma.

Conforme mencionado na seção anterior, a utilização da tecnologia de \textit{blockchain} para a implementação da ferramenta, visa agregar à plataforma algumas características como segurança e durabilidade. Para alcançar tal objetivo, este trabalho fará utilização do conceito de \textit{smart contracts} disponível em plataformas como Ethereum\footnote{https://www.ethereum.org} e Hyperledger\footnote{https://www.hyperledger.org} para a descrição das cláusulas que irão definir as interações entre os participantes.

Dentre as interações que devem estar disponíveis para utilização pelos autores e revisores, podemos destacar (1) a coleta de interessados em colaborar com o desenvolvimento de uma pesquisa em determinada área, (2) garantir que os envolvidos sejam identificados e que o direito sobre o resultado do seu trabalho seja mantido, (3) propor um processo de revisão aberta no qual revisores sejam alocados de acordo com seu nível de conhecimento, (4) garantir o acesso livre ao conteúdo produzido e finalmente (5) especificar um modelo de negócio que seja escalável.

Conforme citado anteriormente neste trabalho, os itens que estão listados como seu objetivo endereçam cada uma das características identificadas como positivas no desenvolvimento de trabalhos científicos, como por exemplo, estímulo a colaboração entre autores e a liberdade de acesso ao conteúdo produzido. Outros aspectos discutidos em \cite{OpenAccessAnalysis2004} também devem ser abordados, sendo o principal deles e também discutido em \cite{InteractivePeerReview2010} a definição de um modelo de negócio capaz de garantir a manutenção da rede. Sob este aspecto pode-se avaliar alternativas como a adotada no PLOS em que os autores (ou instituições) arcam com os custos de publicação, ou a do arXiv na qual foi identificado um conjunto de instituições parceiras, as quais financiam a manutenção do sistema.

Outra característica a ser avaliada com cuidado diz respeito à recompensa dos autores pelas publicações. Seria bastante ingênuo acreditar que autores utilizariam a plataforma apenas por suas características de segurança, acesso livre ao conteúdo, etc. Além disso, autores estão preocupados com a reputação do meio no qual buscam efetuar a publicação. Assim, é importante assegurar que o trabalho submetido pela plataforma seja de alguma forma recompensado em caso de publicação, uma vez que deve ser submetido a um criterioso processo de revisão. A opção por um processo aberto de revisão se dá pelo benefício da transparência \cite{InteractivePeerReview2010}. Este modelo de revisão é utilizado pelo BioMed e se provou positivo já que a identificação dos revisores torna possível que estes sejam cobrados pelo resultado, assim como também são elegíveis a incentivos.

Como resultado deste trabalho, é esperada a implementação de uma plataforma aberta e colaborativa para a escrita de artigos científicos, onde os pares podem colaborar de acordo com sua área de interesse, submetendo seus trabalhos a uma processo de revisão aberta. Essa plataforma adota princípios comprovadamente eficientes para o aumento do impacto dos trabalhos publicados, já que incentiva colaboração e acesso livre ao conteúdo, mantendo o direito sobre o resultado aos autores. Além disso, deverá ser incluída uma forma de verificação no nível de contribuição de cada autor, a fim de evitar que eles sejam incluídos nos trabalhos por quaisquer outras razões \cite{Stealing1993} que não em virtude da sua contribuição.