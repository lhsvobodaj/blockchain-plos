\chapter{Introdução e Motivação}
A quantidade de informação disponível vem crescendo exponencialmente ao longo dos últimos anos. Este crescimento afeta todas as áreas de conhecimento, desde informações livremente publicadas em meios como a Web, até trabalhos acadêmicos os quais são submetidos a processos de revisão antes de se tornarem públicos. Neste contexto, destaca-se o aumento da quantidade de trabalhos científicos. Utilizando como base o banco de dados de publicações da área de computação, o DBLP,\footnote{http://dblp.uni-trier.de/statistics/publicationsperyear.html}, verificamos um crescimento constante de aproximadamente 3\% ao longo dos últimos 5 anos. Em \cite{Online2001} o autor comenta que a quantidade de conteúdo científico de longe excede a capacidade dos pesquisadores de consumi-lo. Além disso, neste mesmo trabalho é ressaltada a importância do acesso livre ao conteúdo publicado, ou \textit{open access publications}.

Existe um interesse crescente da comunidade acadêmica em publicações livremente distribuídas. Esse formato de publicação compreende a disponibilização do conteúdo da pesquisa de maneira livre, principalmente através de meios eletrônicos. A busca por publicações abertas se dá principalmente pelo alcance que podem ter e pela velocidade com que o conteúdo está disponível para o público. Em \cite{Comparing2004} é feita uma comparação do impacto, considerado como sendo o número e citações, entre publicações abertas em relação ao modelo de publicação tradicional. Os autores identificaram uma correlação forte entre o número de citações de um trabalho com o fato de ele estar disponível abertamente para consulta.

Desta forma, em \cite{Online2001}, o autor recomenda como estratégia para maximizar o impacto e acelerar a evolução da Ciência, que autores e organizações facilitem o acesso ao conteúdo científico. Outros estudos sobre o mesmo assunto mostram que após determinado período a partir da data da publicação da pesquisa de forma aberta, mais da metade dos pesquisadores estariam dispostos a pagar pelo conteúdo consumido \cite{Societies2002Misc}. Para evidenciar o impacto de publicações abertas, o estudo conduzido em \cite{OpenAccessImpact2004} verifica que mesmo em áreas de pesquisa como Filosofia, cuja adoção por publicações em meios eletrônicos ocorreu tardiamente em comparação a outras, o número de citações de conteúdo distribuído livremente é 45\% maior em relação a trabalhos disponibilizados da forma tradicional. A partir dessa análise, os autores concluem que a hipótese inicialmente levantada em \cite{Online2001} para Computação se aplica também a outras áreas de pesquisa.

Embora diversas iniciativas de criação de \textit{journals} e repositórios para acesso livre a conteúdo científico tenham sido criados, poucos meios relevantes continuam ativos até a atualidade. Como exemplo, podemos citar arXiv\footnote{https://arxiv.org/}, Public Library of Science (PLOS)\footnote{https://www.plos.org/}, BioMed\footnote{https://www.biomedcentral.com/} e Directory of Open Access Journals (DOAJ)\footnote{https://doaj.org/}, cada um com suas características para submissão, revisão e disponibilização de conteúdo. Em \cite{OpenAccessAnalysis2004} o autor faz uma avaliação sobre 6 aspectos relacionadas a publicações abertas e como elas seriam abordadas em três cenários distintos de publicação: \textit{open access journals}, repositórios abertos e páginas pessoais dos seus autores.

Outro fator que afeta o impacto das publicações é a colaboração entre diferentes pesquisadores. De acordo com \cite{ResearchCollaboration1997}, um colaborador poderia ser definido como alguém que fornece alguma entrada ou contribui com parte da pesquisa. Outra definição seria a de uma pessoa que contribui diretamente para os resultados da pesquisa. Conforme podemos observar, a definição de colaborador não é precisa, mas sim uma convenção do grupo ou área em que os pesquisadores atuam.

Uma das formas de avaliar a colaboração de um trabalho é através dos seus autores \cite{Bibliographical1971}. Embora existam críticas em relação a esse método para avaliação de colaboração, já que em alguns casos autores poderiam ser incluídos apenas por razões sociais e não deveriam receber crédito pelo trabalho \cite{Stealing1993}, verificou-se que o nível de colaboração influencia o impacto do trabalho na comunidade, se considerarmos que trabalhos com múltiplos autores acabam sendo mais citados \cite{Bibliometrics1986}.

Uma das principais motivações para a colaboração é o fato de que a pesquisa moderna demanda conhecimento em diversas áreas. Com frequência, apenas um indivíduo não possui conhecimento suficiente em todas elas. Assim, a opção de colaboração com outros pesquisadores torna-se uma alternativa interessante. Outra perspective em relação à colaboração é o fato de ela poder ser utilizada como uma das métricas de avaliação da qualidade do pesquisador, grupo de pesquisa ou instituição. Como exemplo deste uso, podemos citar a avaliação dos programas de pós-graduação brasileiros feita pela CAPES\footnote{http://capes.gov.br/avaliacao/sobre-a-avaliacao}.

A opção pela publicação de conteúdo científico de maneira aberta e criado de forma colaborativa, traz à tona discussões como garantia de direitos autorais \cite{OpenAccessCopyright2006}, processo de revisão (\textit{peer review}) aberto e modelos de recompensa aos autores pelo seu trabalho. Esses e outros aspectos são discutidos em \cite{OpenAccessAnalysis2004}. Por fim, conforme mencionado anteriormente, com a colaboração entre diversos autores para produção de conteúdo, é necessário um método para determinar o nível de contribuição de cada indivíduo.

Este trabalho tem como objetivo a concepção de uma proposta de plataforma aberta e colaborativa para a elaboração e revisão de trabalhos científicos utilizando como base a tecnologia de \textit{blockchain}. Conforme \cite{UntanglingBlockchain2017},  \textit{blockchain} basicamente consiste em uma rede de nodos na qual não há necessidade de confiança entre todos os participantes, assim como não existe um nodo centralizador ou que exerça qualquer tipo de controle sobre o a rede. Os nodos mantém um conjunto de estados compartilhado e executam transações que alteram e validam esses estados. Assim, pode-se afirmar que \textit{blockchain} é um tipo de estrutura que armazena o histórico de estados e suas mudanças, na qual todos os nodos do sistema concordam com as transações e ordem em que ocorrem. Estas características garantem a esta estrutura as características de integridade, durabilidade e longevidade, já que enquanto a maior parte do poder computacional da rede não estiver sendo utilizada para violá-la, há garantia de que os dados não serão alterados nem perdidos \cite{Bitcoin2008}.

Por padrão, todas as implementações de \textit{blockchain} possuem o conceito de \textit{smart contract} \cite{SmartContract2017}. No caso de uso mais famoso de \textit{blockchain}, o Bitcoin\footnote{https://bitcoin.org}, esse contrato contém o conjunto de cláusulas que descreve como uma transação financeira deve ocorrer. Neste caso, o conceito de transação é definido como sendo um acordo financeiro em que ocorre a troca de valor entre os participantes da rede. Mais recentemente, surgiram implementações de \textit{blockchain} que permitem aos seus participantes definirem seus próprios contratos, de forma que eles sejam executados na rede quando uma transação é iniciada.

Através da utilização das tecnologias de \textit{blockchain} e contratos inteligentes, esta dissertação fará uma proposta de plataforma que ofereça as garantias necessárias para que pesquisadores possam desenvolver trabalhos colaborando entre si, assegurando as garantias de autoria e níveis de contribuição. Além disso, essa plataforma deverá prover suporte a um modelo de revisão aberto onde os revisores são conhecidos e podem inclusive ser recompensados pelas suas revisões.