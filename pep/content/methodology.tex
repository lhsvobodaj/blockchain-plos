\chapter{Metodologia}
A metodologia de trabalho prevista compreende o estudo avançado sobre alguns temas. Inicialmente, será feita uma análise bibliográfica extensiva sobre os requisitos para adoção de um padrão de livre acesso à conteúdo científico. O objetivo desta etapa, além de verificar as vantagens desse modelo, é garantir que autores e revisores tenham seus direitos mantidos, evitando assim possíveis penalidades ou perdas de garantias para ambos. Em seguida, avaliaremos métodos ara execução de \textit{peer review} abertas, conforme citado em \cite{InteractivePeerReview2010}. Da mesma forma como no estudo anterior, o objetivo é entender os modelos possíveis para implementação de um modelo de revisão pelos pares aberta e quais as vantagens e desvantagens de cada abordagem.

Uma vez que os conceitos fundamentais para execução do trabalho tenham sido entendidos, será iniciada a avaliação sobre das plataformas de \textit{blokchain} disponíveis para utilização. Nesta etapa será realizada uma avaliação simplificada para verificação das funcionalidades suportadas em cada plataforma. De modo concomitante, será conduzida uma avaliação do suporte à criação de \textit{smart contracts} em cada plataforma. Esse etapa é fundamental uma vez que é necessário garantir que a linguagem a ser utilizada para definição do contrato deve ter expressividade suficiente para garantir que todas as clausulas do contrato possam ser definidas. Por fim, em virtude dos custos associados na utilização de \textit{blockchain} para armazenamento de dados, avaliaremos opções fora da \textit{chain} para persistência dos dados. Por exemplo, nesta etapa avaliaremos ferramentas como IPFS\footnote{https://ipfs.io/}.

A próxima etapa compreende a implementação de um protótipo cujo objetivo é fornecer suporte ao processo de coleta de colaboradores e desenvolvimento do trabalho de maneira colaborativa entre os participantes. Considerando que os envolvidos no desenvolvimento do trabalho tenham finalizado seu esforço inicial, deve ser possível efetuar a submissão para revisão pelos pares. Nesta fase é importante verificar que os revisores sejam selecionados de forma adequada para que o resultado da revisão agregue e garanta mais qualidade ao trabalho final. É importante que o protótipo da plataforma atenta aos requisitos citados na seção anterior deste documento. Esta fase também compreende o suporte ao acesso livre ao conteúdo, bem como a implementação de um modelo de negócios sustentável para garantir a evolução da plataforma, bem como garantir a recompensa para os envolvidos em todo o processo.

Por fim, após as etapas de criação do protótipo e avaliação, será desenvolvida a redação da dissertação e de um artigo científico que deve ser submetido como parte dos requisitos para obtenção do grau de mestre em computação. O meio de publicação do artigo (conferência ou \textit{journal} deverá ser definido ao longo do desenvolvimento da dissertação.