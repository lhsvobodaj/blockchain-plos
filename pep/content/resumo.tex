\keyword{blockchain}
\keyword{escrita colaborativa}
\keyword{sistema cooperativo}

\begin{abstract}
A liberdade para criação e acesso à informação evoluiu drasticamente com o advento da Web. Embora essa evolução possa ser observada em todas as áreas de conhecimento, o acesso ao conteúdo científico ainda está sob controle de algumas instituições. Mesmo a elaboração de trabalhos científicos sendo ser muitas vezes fomentada por órgãos do estado, o acesso a esse conteúdo fica restrito a grupos ou indivíduos que se dispõem a pagar por ele. Uma das iniciativas mais influentes que tenta reverter essa situação é o Public Library of Science (PLOS). O PLOS é uma entidade cujo objetivo é tornar livre o acesso às publicações acadêmicas, através de um modelo no qual os autores arcam com os custos de revisão e possivelmente publicação do trabalho. Este mecanismo por si só não garante a legitimidade das revisões já que os revisores continuam sendo anônimos, além de não reverter qualquer tipo de financiamento para os autores. Sendo assim, este trabalho tem como objetivo o desenvolvimento de uma plataforma aberta para edição e revisão de trabalhos científicos, utilizando como base para isso a tecnologia de \textit{blockchain}. O emprego de blockchain para este fim traz consigo os benefícios inerentes a sua utilização, tais como durabilidade, confiabilidade e longevidade, já que os dados inseridos na {\em chain} não podem ser alterados. Outro ponto importante da utilização desta tecnologia é a descentralização, o que facilita a contribuição entre os participantes da rede.
\end{abstract}
