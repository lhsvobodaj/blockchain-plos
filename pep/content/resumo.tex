\keyword{blockchain}
\keyword{colaboração}
\keyword{\textit{open access}}

\begin{abstract}
A liberdade para criação e acesso à informação evoluiu drasticamente com o advento da Web. Embora essa evolução possa ser observada em todas as áreas de conhecimento, o acesso ao conteúdo científico ainda está sob controle de algumas instituições. Uma reação a esse controle são as iniciativas que fornecem acesso livre ao conteúdo publicado. Como exemplo de tais iniciativas podemos citar o arXiv, PLOS e BioMed. A opção por publicações abertas levanta a discussão sobre como manter os direitos autorais sobre o trabalho, como efetuar um processo de revisão justo e como recompensar os autores pelo trabalho, baseado no nível de colaboração de cada indivíduo ou instituição. Desta forma, este trabalho tem como objetivo o desenvolvimento de uma plataforma aberta e colaborativa para escrita e revisão de trabalhos científicos, utilizando como base para isso a tecnologia de \textit{blockchain}. O emprego de blockchain para este fim traz consigo os benefícios inerentes a sua utilização, tais como durabilidade, confiabilidade e longevidade, já que os dados inseridos na {\em chain} não podem ser alterados. Outro ponto importante da utilização desta tecnologia é a descentralização, o que facilita a contribuição entre os participantes.
\end{abstract}
